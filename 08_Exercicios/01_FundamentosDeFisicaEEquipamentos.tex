\documentclass[11pt,a4paper]{article}
\usepackage[top=3cm, bottom=2cm, left=2cm, right=2cm]{geometry}
\usepackage[utf8]{inputenc}
\usepackage{amsmath, amsfonts, amssymb}
\usepackage{siunitx}
\usepackage[brazil]{babel}
\usepackage{graphicx}
\usepackage[margin=10pt,font={small, it},labelfont=bf, textfont=it]{caption}
\usepackage[dvipsnames, svgnames]{xcolor}
\DeclareCaptionFont{MediumOrchid}{\color[svgnames]{MediumOrchid}}
\usepackage[pdftex]{hyperref}
\usepackage{natbib}
\bibliographystyle{plainnat}
\bibpunct{\textcolor{MediumOrchid}{\textbf{[}}}{\textcolor{MediumOrchid}{\textbf{]}}}{,}{s}{}{}
\usepackage{color}
\usepackage{footnote}
\usepackage{setspace}
\usepackage{booktabs}
\usepackage{multirow}
\usepackage{subfigure}
\usepackage{fancyhdr}
\usepackage{leading}
\usepackage{indentfirst}
\usepackage{wrapfig}
\usepackage{mdframed}
\usepackage{etoolbox}
\usepackage[version=4]{mhchem}
\usepackage{enumitem}
\usepackage{caption}
\usepackage{titlesec}
\usepackage{tcolorbox}
\usepackage{tikz}
\usepackage{LobsterTwo}
\usepackage[T1]{fontenc}
\usepackage{fontspec}
\usepackage{txfonts}
\usepackage[bottom]{footmisc}
\tcbuselibrary{skins,breakable}
\sisetup{output-decimal-marker={.}}

\makeatletter
\def\footnoterule{\kern-3pt\color{MediumOrchid}\hrule\@width0.6\textwidth height 0.8pt\kern2.6pt}
\makeatother

\renewcommand{\footnotelayout}{\itshape\color{MediumOrchid}}

\AtBeginEnvironment{equation}{\fontsize{13}{16}\selectfont}


\titleformat{\section}{\LobsterTwo\huge\color{CarnationPink}}{\thesection.}{1em}{}
\titleformat{\subsection}{\LobsterTwo\huge\color{CarnationPink}}{\thesubsection}{1em}{}
\titleformat{\subsubsection}{\bf\LobsterTwo\Large\color{MediumOrchid}}{\thesubsubsection}{1em}{}


\DeclareCaptionLabelFormat{figuras}{\textcolor{DarkTurquoise}{Figura \arabic{figure}}}
\captionsetup[figure]{labelformat=figuras}

\makeatletter
\renewcommand\tagform@[1]{\maketag@@@{\color{CarnationPink}(#1)}}
\makeatother

\renewcommand{\theequation}{Eq. \arabic{equation}}
\renewcommand{\thefigure}{Fig. \arabic{figure}}
\renewcommand{\thesection}{\textcolor{CarnationPink}{\arabic{section}}}

\setlist[itemize]{label=\textcolor{CarnationPink}{$\blacksquare$}}

\setlist[enumerate]{label=\textcolor{CarnationPink}{\arabic*.}, align=left, leftmargin=1.5cm}


\newcounter{exemplo}

\NewDocumentEnvironment{exemplo}{ O{} }{%
\allowbreak
\setlength{\parindent}{0pt}
  \begin{mdframed}[
  leftline=true,
  topline=false,
  rightline=false,
  bottomline=false,
  linewidth=2pt,
  linecolor=CarnationPink,
  frametitlerule=false,
  frametitlefont=\LobsterTwo\large\color{CarnationPink},
  frametitle={\color{CarnationPink}\LobsterTwo\large #1},
  ]
}{%
  \end{mdframed}
}

\setlength{\fboxsep}{5pt}
\setlength{\fboxrule}{1.5pt}
\usepackage{float}
\renewcommand{\thefootnote}{\alph{footnote}}
\usepackage{url}
\hypersetup{
	colorlinks=true,
	linkcolor=DarkTurquoise,
	filecolor=DarkTurquoise,      
	urlcolor=DarkTurquoise,
	citecolor=DarkTurquoise,
	pdftitle={Especialista em Física da Radioterapia}
}
\pagestyle{fancy}
\fancyhf{}
\renewcommand{\headrulewidth}{0pt}
\rfoot{\color{DarkTurquoise}\thepage \\ \LobsterTwo{\small\textcolor{CarnationPink}{@defDalila}}}

\title{\LobsterTwo\Huge{Exercícios}}
\author{\LobsterTwo\Large{Fundamentos de Física E Equipamentos Geradores de Radiação}\nocite{*}}
\date{\LobsterTwo{Dalila Mendonça}}
\begin{document}
	\maketitle

\begin{exemplo}[Fundamentos de Física ]

    \textcolor{MediumOrchid}{\LobsterTwo\textbf{Grandezas e Unidades}}

    \begin{itemize}
        \item As unidades de radioatividade são o Becquerel (Bq) e Curie (Ci). 1 Bq = 1 decaimento/seg. 1 curie = 3,7 x $10^{10}$ decaimentos/seg; Portanto, 1 mCi = 3,7 x $10^7$ Bq.
        
        \item O roentgen é a unidade de exposição. A exposição é a medida da ionização causada pelos fótons no ar. É definido como a carga de um sinal liberada por unidade de massa. 1R = $2.58 \times 10^{-4}$ C/kg na temperatura padrão (\ang{20}C) e pressão (760 mmHg).
        
        \item A unidade SI para dose absorvida é o gray (Gy) tal que 1 Gy = 1 J/kg. Rad é a antiga unidade de dose absorvida onde, 1 Gy = 100 rad ou 1 rad = 1 cGy.

        \item O Sv e o rem medem o efeito da dose no corpo (utilizados para fins de proteção radiológica). O que inclui o efeito biológico de diferentes tipos de radiação como nêutrons (através da dose equivalente), ou o aumento da sensibilidade de diferentes órgãos (através da dose absorvida). A unidade SI é o Sv. 1 Sv = 1 J/kg e 1 mSv = 100 mrem.

        \item O elétron-volt é definido como a energia cinética adquirida por um elétron inicialmente em repouso passando por uma diferença de potencial de 1 V. É uma unidade de energia com 1 eV = $1,602 \times 10^{-19}$ Joules (J) e 1 amu = 931 MeV. O amu representa um doze avos da massa de um núcleo de carbono-12.

        \item A energia de um fóton é igual à sua frequência (em Hz) multiplicada pela constante de Planck ($h = 6.26 \times 10^{-34} $) J s. A frequência ($\nu$) está relacionada ao comprimento de onda ($\lambda$) por  $c = \lambda \cdot \nu$. A velocidade (c) é a velocidade da luz que vale aproximadamente $3 \times 10^{8}$ m/s.
        
        \item Os fótons se propagam como ondas eletromagnéticas, descritas pelo espectro eletromagnético. Este espectro define regiões com base em sua energia (e, portanto, comprimento de onda ou frequência). Da energia mais baixa (e frequência mais baixa, comprimento de onda maior) o espectro começa com ondas de rádio, depois micro-ondas,i nfravermelho, luz visível, ultravioleta, raios X e raios gama.

        \item A atividade específica de uma fonte é definida como a atividade de uma amostra (A) dividida por sua massa (m). A unidade SI para determinados atividade é Bq/kg. Também pode ser dado em termos de  Ci/g. Uma atividade específica mais alta permite o desenvolvimento de uma fonte menor para ser usado nos tratamento com radiação. Por exemplo, a atividade específica da fonte de \ce{^{60}Co} é de 200 Ci/g e pode conter 6.000 Ci a 7.000 Ci em uma fonte de \ce{^{60}Co} de 1,5 a 2,0 cm de diâmetro.
        
    \end{itemize}

    \textcolor{CarnationPink}{Estrutura Atômica E Nuclear}
    \begin{itemize}
        \item A constante de Planck (h) é dada como $h = 6.62 \times 10^{-34} \; J s$. Um fóton não possui nenhuma massa ou tem qualquer carga, mas possui energia que está relacionada com a frequência ($\nu$) do fóton pelo seguinte equação, $E = h\nu$.
        
        \item Sabemos que $c = \lambda \cdot \nu$, onde $\lambda$ é o comprimento de onda e c é a velocidade da luz ($3 \times 10^{8} \; m/s$). Ao usar essa relação, podemos determinar a frequência $\nu$ e então usar $E = h \nu$ para calcular a energia de um fóton.
        
        \item A constante de Avogadro é $N_A = 6.022 \times 10^{23}$, é o número de átomos em um mol de uma substância.
        
        \item A carga de um elétron é $q = 1.6 \times 10^{-19}$ C. O Coulomb (C) é a unidade de carga elétrica e representa um carga total de $6.24 \times 10^{18}$ elétrons. Na física clássica, a menor unidade de carga negativa é um elétron com uma carga de "-1q", enquanto o próton tem a menor unidade de carga positiva, "+1q". 

        \item A energia de ligação do elétron é determinada pela atração Coulombiana entre o núcleo (contendo prótons) e os elétrons orbitais ($e^{-}$). A energia de ligação do elétron aumenta à medida que Z aumenta e diminui à medida que a distância do núcleo aumenta. A energia de ligação de um elétron é a energia mínima necessária para arrancar o elétron do átomo.
        
        \item As camadas eletrônicas são nomeadas a partir da camada mais interna do átomo, através de letras ou números. A camada mais interna é a camada “K” seguido pelas camadas L, M e N, respectivamente. O número máximo de elétrons que são permitidos em uma camada é 2n\textsuperscript{2}, onde n é o número da camada do orbital, indo de 1 até a camada mais externa, por exemplo K = 1, L = 2, e assim sucessivamente. No entanto, a camada mais externa de um átomo comumente chamada de camada de valência só é capaz de transportar oito elétrons.
        
        \item Um elétron tem uma carga de -1q e uma energia de repouso de 0,511 MeV. A massa de um elétron é 1/1.836 em comparação com a massa de um próton. O próton tem uma carga de +1q e uma energia de repouso de 938 MeV. O nêutron tem uma carga de 0q e uma energia de repouso de 939 MeV.
        
        \item Para um dado símbolo atômico, o número de prótons é igual a “Z” que também é igual ao número de elétrons quando está em equilíbrio; “A” é o número de massa que também é igual a soma do número de prótons + nêutrons e também é referido como o número de núcleons. O número de nêutrons é igual a A – Z.
        
        \item Comparando a força forte, força fraca, força eletromagnética e força gravitacional, a mais fraca é a força gravitacional que é a força que atrai as massas, mas na escala do mundo atômico é desprezível. Em seguida, a força fraca que é responsável pelo decaimento nuclear. A força eletromagnética existe entre todas as partículas que possuem carga elétrica. A força forte é responsável por manter o núcleo unido e atua em alcances muito curtos.
        
        \item Um elemento é definido por seu número atômico Z, que representa o número de prótons, igual ao número de elétrons e dá ao elemento suas propriedades químicas. O mesmo elemento pode ter diferentes números de nêutrons, mas o mesmo número de prótons, estes são referidos como isótoPos. Um isótoNo é o oposto, são núcleos com o mesmo número de nêutrons, mas diferentes números de prótons. IsóBARos são núcleos que possuem o mesmo número de núcleons (por exemplo, sete prótons e oito nêutrons, ou seis prótons e nove nêutrons), ou seja, o mesmo número atômico. Um isômero é o mesmo núcleo (o mesmo número de prótons, nêutrons e o mesmo número atômico), mas é um estado excitado, geralmente instável do núcleo.
        
        \item Se um átomo é ionizado, uma vacância pode ser criada em um orbital eletrônico interno. Um elétron em um orbital externo preencherá a vacância e um fóton com energia igual à diferença de energia dos dois orbitais será criado. Essa energia é característica do elemento particular envolvido.
        
        \item Um elétron Auger é um processo alternativo e concorrente aos raios X característicos que às vezes podem ocorrer. Neste caso, em vez de um raio-X característico ser emitido pelo átomo, o excesso de energia é transferido para outro elétron em órbita, que é ejetado do átomo. Isso é conhecido como um elétron Auger.
        
        \item A Fissão Nuclear é a divisão de um grande núcleo instável em partes mais estáveis. A fusão combina dois ou mais núcleos mais leves em apenas um núcleo, como ocorre no sol quando o hidrogênio se funde para formar o hélio.
        
        \item A energia de ligação por núcleon tem um máximo próximo de A = 56, e a energia de ligação média (exceto para elementos leves) é de cerca de 8 MeV por núcleon. Tanto a fissão (para A $>$ 56) quanto a fusão (para A $<$56) ocorrem para permitir que os núcleos se aproximem de um estado mais estável que esteja mais próximo da energia máxima de ligação por núcleon.
        
        \item A produção fluorescente $\omega$ é definida como a probabilidade de um átomo produzir radiação característica em vez de um elétron Auger. As seguintes relações são notadas:
            \begin{itemize}
                \item Alto-Z → grande $\omega$ → emissão de radiação característica mais provável;
                \item Baixo-Z → pequeno $\omega$ → emissão de elétrons Auger mais provável.
            \end{itemize}
           
       \item Os raios X e os raios gama são fótons em forma de energia eletromagnética, mas a diferença entre ele é que os raios gama são definidos como sendo de origem nuclear e os raios X de origem atômica.
       
       
    \end{itemize}

    \textcolor{MediumOrchid}{\LobsterTwo\textbf{Radioatividade}}

    \begin{itemize}
        \item Em um decaimento radioativo, uma transição isomérica significa que o núcleo está mudando de estado de energia, sem alterar o número de prótons e nêutrons dentro dele. Uma maneira pela qual o núcleo pode liberar energia é chamada de emissão gama. Na emissão gama, o núcleo libera o excesso de energia emitindo diretamente um ou mais raios gama do núcleo. Um método alternativo de liberação de energia é a conversão interna, onde um ou mais elétrons orbitais são emitidos. Se um elétron da camada interna for emitido, ocorrerá o preenchimento da camada e resultará na liberação de raios-X característicos e/ou elétrons Auger.
       
       \item A constante de decaimento está relacionada com a meia-vida: $\lambda = \ln 2 / T_{1/2}$.
       
       \item Quando a constante de decaimento do núcleo filho é muito maior que a do núcleo pai ocorre um equilíbrio chamado de equilíbrio secular. Esse equilíbrio é caracterizado por um aumento gradual da atividade do núcleo filho até atingir o nível do núcleo pai. Depois de estabelecido o equilíbrio secular, a atividade do núcleo filho é aproximadamente igual à atividade do núcleo pai. É o caso do \ce{^{226}Ra} (Rádio) e seu filho \ce{^{222}Rn} {Radônio}. A constante de decaimento é inversamente proporcional à meia-vida; portanto, para o equilíbrio secular, a meia-vida do núcleo filho deve ser muito mais curta que a meia-vida do núcleo pai. Se a constante de decaimento do núcleo filho for maior que a constante de decaimento do pai, mas não muito maior (ligeiramente maior), então o equilíbrio transiente é alcançado. No equilíbrio transiente, há um crescimento inicial do núcleo filho até que eventualmente exceda a atividade do núcleo pai. Depois desse ponto, a atividade do núcleo filho segue a atividade do núcleo pai, sempre excedendo o núcleo pai por um pequeno valor. Isso é exemplificado pelo \ce{^{99}Mo} (Molibdênio) e seu núcleo filho filha \ce{^{99m}Tc} (Tecnécio). 
       
       \item Os radionuclídeos que têm uma razão n/p baixa tendem a aumentar a razão convertendo um próton em um nêutron. No decaimento beta mais, ou emissão de pósitrons, um próton se converte em um nêutron emitindo um ``elétron positivo'' (pósitron ou beta mais) e um neutrino. Um processo concorrente com o decaimento beta mais é a captura eletrônica. Neste processo, um dos elétrons orbitais internos (geralmente da camada K) é capturado pelo núcleo. O núcleo se reorganiza e transforma um próton em um nêutron para alcançar um novo estado estável. Depois que um elétron é capturado pelo núcleo no decaimento de captura de elétrons, raios X característicos ou elétrons Auger serão produzidos.
       
	   \item A proporção de nêutrons para prótons (n/p) é 1 para elementos atômicos onde Z é menor ou igual a 20. Se Z for maior que 20, a proporção aumenta com Z. Os nêutrons adicionais são necessários para ajudar a manter os núcleos estáveis e para compensar repulsão eletrostática entre os prótons.
	   
	   \item A transição isomérica ocorre quando um decaimento radioativo anterior deixa o núcleo filho em um estado metaestável que então transita para o estado fundamental. O núcleo filho permanece em um estado excitado por um curto período de tempo. A única diferença entre o estado metaestável e o estado fundamental final é uma diferença de energia, portanto os dois estados são chamados de isômeros. Um exemplo disso é o \ce{^{99m}Tc} onde o \ce{^{99}Mo} decai via decaimento beta para \ce{^{99m}Tc}. Este estado metaestável, tem uma meia-vida de 6 horas. Em seguida, o \ce{^{99m}Tc} decai por emissão gama, liberando um fóton de 141 keV, que é usado para geração de imagens em tomografia computadorizada por emissão de fóton único (SPECT).
	   
	   \item Um decaimento isobárico é qualquer decaimento radioativo no qual o núcleo original (ou núcleo pai) e o novo (ou núcleo filho) contém o mesmo número total de núcleons (prótons + nêutrons). Quando isso ocorre, o núcleo pai e o núcleo filho são referidos como isóbaros um do outro. Ambos os tipos de decaimento beta são isobáricos.
	   
	   \item O núcleo é formado por prótons e nêutrons, mas a massa do núcleo será sempre menor que a soma das massas individuais dos prótons e nêutrons que o constituem. A diferença na massa é chamada de defeito de massa (déficit de massa), e a energia necessária para separar o núcleo em suas partículas constituintes é chamada de energia de ligação do núcleo e é dada pela equação de Einstein: $E = \Delta mc^2$, onde $\Delta m$ é a diferença de massa do núcleo e suas partículas constituintes (prótons + nêutrons).
	   
	   \item Os dois tipos de decaimento beta são o decaimento beta menos e o decaimento beta mais. O decaimento beta é o processo pelo qual um núcleo radioativo ejeta um elétron carregado negativamente (beta menos ou $\beta^-$) ou um pósitron carregado positivamente (beta mais ou $\beta^+$). Um elétron que é ejetado é diferenciado de um elétron orbital usando o termo $\beta^-$ em vez de “e”, que se refere a um elétron orbital. Ambas as formas de decaimento beta são isobáricas, pois o número atômico total não muda.
	   
	   \item No decaimento beta menos, um nêutron é convertido em um elétron (também conhecido como partícula $\beta^-$), um próton e um antineutrino. Este tipo de decaimento é mais comum com um núcleo que tem uma razão n/p $>$ 1. A equação geral para um beta menos decaimento é:
		$$\ce{n -> p + \beta- + \bar{\nu}}$$
	   A energia liberada é dada pela diferença na energia de ligação do núcleo pai para o núcleo filho, menos a massa do elétron. Esta energia é compartilhada como energia cinética das partículas de saída, com em média um terço indo para a partícula $\beta$, e o restante para o anti-neutrino ($\bar{\nu}$). O anti-neutrino não carrega nenhuma carga elétrica e interage fracamente. Haverá um aumento no número atômico Z do elemento filho em uma unidade depois que ocorrer o decaimento beta menos.

       \item O decaimento beta mais também é conhecido como emissão de pósitrons e ocorre quando um radionuclídeo tem uma baixa proporção de nêutrons em relação aos prótons (n/p). No decaimento beta mais, um próton (p) é convertido em um pósitron (também conhecido como partícula $\ce{\beta+}$), um nêutron (n) e neutrino ($\nu$). A equação geral para beta mais decaimento é:
       
		$$\ce{p -> n + \beta+ + \nu}$$

	   No decaimento beta mais, há uma diminuição no número atômico do elemento filho em uma unidade. Depois que uma partícula beta mais é produzida, ela eventualmente interagirá com um elétron do meio fazendo com que ambas as partículas sejam aniquiladas, produzindo dois fótons (raios gama), cada um com energia de 0,511 MeV e viajando em direções opostas. Por causa disso, há um limiar na diferença entre a energia de ligação do pai e do filho de 1,022 MeV para que um radionuclídeo sofra emissão de pósitrons.

	   \item Um processo concorrente ao decaimento beta mais é a captura eletrônica. Esse processo ocorre quando o núcleo captura um elétron orbital, tipicamente da camada K, e o próton é convertido em um nêutron (n) e neutrino ($\nu$). É um processo que prevalece para elementos mais pesados. A equação geral para a captura de elétrons é:
	   $$\ce{p + e -> n + \nu}$$
	   Como a captura de elétrons deixa uma vacância na camada eletrônica, essa vacância poderá ser preenchida por um elétron de uma camada mais externa, causando a liberação de raio-X característico ou elétron Auger. Como a captura eletrônica geralmente envolve o elétron orbital da camada K, ela é frequentemente chamada de captura K. Da mesma forma que o decaimento beta mais, haverá uma diminuição no número atômico do elemento filho em uma unidade depois que um núcleo sofrer captura eletrônica.

	   \item O decaimento alfa é um tipo de decaimento radioativo em que um núcleo emite uma partícula alfa. Uma partícula alfa tem a mesma estrutura nuclear que um núcleo de hélio. Uma partícula alfa é geralmente designada pelo símbolo \ce{^{4}_{2}He}. Os radionuclídeos que têm um alto Z (geralmente Z > 82) decaem com mais frequência pela emissão de uma partícula alfa. Um exemplo de decaimento alfa é quando o rádio (Ra) sofre decaimento para
	   radônio (Rn):
	   $$\ce{^{226}_{88}Ra -> ^{222}_{86}Rn + ^{4}_{2}He + Energia}$$
    \end{itemize}

	\textcolor{MediumOrchid}{\LobsterTwo\textbf{Decaimento Exponencial}}

	\begin{itemize}
		\item A constante de decaimento, $\lambda$, é uma constante de proporcionalidade que relaciona o número de desintegrações de átomos radioativos por unidade de tempo com o número de átomos presentes. A relação entre o número de átomos restantes (N), para o número de átomos iniciais (N0) é relacionada pela equação exponencial:
		$$N(t) = N_0 e^{- \lambda t}$$

		\item O tempo necessário para que a atividade ou o número de átomos radioativos decaia para metade do valor inicial é a definição de tempo de meia-vida. (Meia-vida) T\textsubscript{1/2} = ln(2)/$\lambda$ = (0,693)/$\lambda$. Após n meias-vidas, a atividade ou o número de átomos radioativos é reduzido a (1/2)\textsubscript{n} do valor inicial.
		
		\item O tempo de vida média é o tempo de duração médio  que um átomo radioativo permanece em sua amostra. O tempo de vida média ($\tau$) é igual a 1,44 x a meia-vida (T\textsubscript{1/2}). Também é equivalente ao tempo que uma amostra levaria para decair se continuasse a decair em sua taxa de decaimento inicial. Como tal, é útil no cálculo da dose fornecida por implantes permanentes.
		
		\item A atividade específica de um radionuclídeo é definida como sendo a atividade do radionuclídio por unidade de massa. A atividade específica é independente da massa do radionuclídeo, e tem um valor fixo independente do tempo. As unidades de atividade específica são Ci/g ou Bq/kg.

	\end{itemize}
\end{exemplo}


\begin{exemplo}[Equipamento Gerador de Radiação]

	\textcolor{MediumOrchid}{\LobsterTwo\textbf{Produção de Raios-X}}
    \begin{itemize}
        \item Os raios X são produzidos por bremsstrahlung e emissão de raios X característica. Feixes de raios X úteis em imagens e terapia são tipicamente todos bremsstrahlung, exceto em mamografia, onde os raios X característicos são desejáveis.
        
		\item Os raios X de Bremsstrahlung resultam da interação Coulombiana entre o elétron incidente e os núcleos do material alvo. O elétron incidente é desacelerado, perdendo energia cinética na forma de fótons bremsstrahlung (também chamada de perda radiativa).
		
		\item Os raios X característicos resultam das interações Coulombianas entre os elétrons incidentes e os elétrons orbitais atômicos do material alvo (perda por colisão). O elétron orbital é ejetado de sua camada e um elétron de uma camada de nível superior preenche a vacância. A diferença de energia entre as duas camadas pode ser emitida do átomo na forma de um fóton característico (raio X característico) ou transferida para outro elétron orbital que é ejetado do átomo como um elétron Auger.
		
		\item Os raios X de Bremsstrahlung apresentam um espectro de energia contínuo. A energia máxima dos raios X é igual à energia do elétron incidente. A energia do elétron corresponde ao pico de tensão de aceleração. A energia mais provável dos raios-X é cerca de um terço da energia máxima. 
		
		\item Os raios X característicos apresentam um espectro de energia discreto, correspondendo à diferença de nível de energia entre camadas atômicas envolvidas na transição de elétrons. Os fótons de raios X produzidos por um tubo de raios X são a combinação dos dois tipos de raios X emitidos. Tem uma distribuição contínua de energias para os fótons bremsstrahlung sobrepostos com radiação característica em energias discretas.
		
		\item Os raios X usados em oncologia de radiação são geralmente classificados como:
		
			\begin{enumerate}[label=\alph*)]
				\item \textbf{Grenz ray therapy}: tratamento que utiliza raios X de baixíssima energia (abaixo de 20 kV). Não está mais em uso.
				
				\item \textbf{Terapia de contato}: opera de 40 a 50 kV e facilita a irradiação de lesões acessíveis em distância muito curtas da fonte à superfície (SSD) (2 cm ou menos). Um filtro de alumínio de 0,5 a 1,0 mm de espessura é geralmente interposto no feixe para absorver o componente muito suave do espectro de energia. É útil para tumores não mais profundos do que 1 a 2 mm.
				
				\item \textbf{Terapia superficial}: tratamento com raios X variando de 50 a 150 kV. Espessuras variadas de filtração (geralmente alumínio de 1 a 6 mm) são adicionadas para endurecer o feixe em um grau desejado. A SSD normalmente varia entre 15 e 20 cm. Útil para irradiar tumores confinados a cerca de 5 mm de profundidade.
				
				\item \textbf{Terapia de ortovoltagem}: tratamento com raios X variando de 150 a 500 kV e filtrados com cobre de 1 a 4 mm. A SSD normalmente tem 50 cm. Útil para tumores com menos de 2 a 3 cm de profundidade.
				
				\item \textbf{Terapia de Supervoltagem}: terapia de raios X na faixa de 500 a 1.000 kV, filtrada com cobre de 4 a 6 mm.
				
				\item \textbf{Terapia de Megavoltagem}: terapia de raios X acima de 1 MV.
			\end{enumerate}

		\item Na faixa de energia de quilovoltagem, os raios X são produzidos uniformemente no que diz respeito à direção dos fótons produzidos, ou seja, são emitidos em todas as direções; A blindagem ao redor do alvo produz um feixe útil de raios X a 90° em relação à direção de aceleração dos elétrons. Na faixa de energia de megavoltagem (1–50 MV), a maioria dos fótons é produzida na direção da aceleração do elétron (direção  à frente 0°).
		
		\item O rendimento fluorescente (produção fluorescente) $\omega$ dá a proporção de fótons fluorescentes (característicos) emitidos por vacância em uma camada comparado ao número de elétrons Auger. $\omega$ varia de zero, para átomos de baixo Z,  0,5 para cobre (Z = 29) até 0,96 para átomos de alto Z com vacâncias  na camada K, que são as fontes mais proeminentes de raios-X característicos.
		
		\item A eficiência da produção de raios X é proporcional ao número atômico (Z) do alvo e à energia dos elétrons. A eficiência é inferior a 1\% para tubos de raios X operando a 100 kVp (99\% da energia de entrada é convertida em calor). A eficiência melhora consideravelmente para feixes aceleradores de megavoltagem (30\%–95\%, dependendo da energia).
		
		\item Um tubo de raios X típico consiste em um invólucro de vidro altamente evacuado, um cátodo (eletrodo negativo, filamento de tungstênio), que produz os elétrons, e um ânodo (eletrodo positivo, alvo de tungstênio ligado a uma haste de cobre grossa).
		
		\item A escolha do tungstênio para filamento (cátodo) e alvo (ânodo) é baseada em seu alto ponto de fusão (3.370°C), devido ao calor absorvido, e um alto número atômico (Z = 74) para aumentar a eficiência na produção dos raios-X.
		
		\item Em um tubo de raios-x, um alvo fino tem uma espessura muito menor que o alcance dos elétrons R para uma determinada energia, enquanto a espessura de um alvo grosso é da ordem de R.
		
		\item O objetivo da filtragem adicional em um tubo de raios-X é aumentar a proporção de raios X de alta energia no feixe, absorvendo os componentes de energia mais baixa do espectro. Assim, o feixe transmitido “endurece” (ou seja, atinge maior energia média e, portanto, maior poder de penetração). Outra forma de melhorar o poder de penetração do feixe é aumentando a voltagem através do tubo. Como a intensidade total do feixe diminui com o aumento da filtração e aumenta com a voltagem, é necessária uma combinação adequada de voltagem e filtração para atingir o endurecimento desejado do feixe, bem como uma intensidade aceitável para realizar uma imagem.
		
		\item Uma Camada Semi-Redutora (HVL) é definida como a espessura do material necessária para reduzir a intensidade de um feixe para metade do seu valor inicial. Uma quantidade relacionada à essa espessura é o coeficiente de atenuação linear $\mu$, onde HVL = 0.693/$\mu$.
		
		\item Ao atravessar um absorvedor os fótons de menor energia do feixe polienergético de raios-X serão preferencialmente removidos do feixe enquanto atravessam este meio. A mudança do espectro de raios-X para energias efetivas mais altas à medida que o feixe atravessa a matéria é chamada de endurecimento do feixe. Para feixes polienergéticos, isso faz com que a segunda camada semi-redutora (HVL) seja maior que a primeira HVL.
		
		\item Uma vez que os raios X são produzidos em várias profundidades no alvo em um tubo de raios-x, eles sofrem diferentes quantidades de atenuação no alvo. Há maior atenuação para os raios X vindos de profundidades maiores do que aqueles próximos à superfície do alvo. Consequentemente, a intensidade do feixe de raios-X diminui do cátodo para a direção do ânodo do feixe. Essa variação no feixe de raios-X é chamada de efeito anódico (heel effect). O efeito é particularmente pronunciado em tubos de diagnóstico devido à baixa energia de raios-X e ângulos de alvo acentuados. O problema pode ser minimizado usando um filtro compensador para fornecer atenuação diferencial ao longo do feixe para compensar o efeito anódico e melhorar a uniformidade do feixe.
		
		\item O tamanho da área alvo da qual os raios X são emitidos é chamado de ponto focal. O tamanho do ponto focal depende do tamanho do filamento de tungstênio do cátodo. Na radiologia diagnóstica, o ponto focal deve ser o menor possível para produzir imagens radiográficas nítidas. Mas os pontos focais menores geram mais calor por unidade de área do alvo e, portanto, limitam as correntes e a exposição. Os tubos de diagnóstico geralmente têm dois filamentos separados para fornecer “foco duplo”, um filamento pequeno e um grande para imagens pequenas e grandes.
		
		\item Todos os feixes de raios-X têm alguma filtragem; isto é tipicamente devido à absorção no alvo, pela parede do tubo e pelo ar. Essa filtragem é chamada de filtragem inerente.
		
		\item A diferença entre a corrente do tubo e a corrente do filamento é que a corrente de filamento é o fluxo de elétrons através do filamento presente no cátodo. Esse fluxo de elétrons aquece o filamento e a emissão termiônica acontece quando a corrente do filamento é grande o suficiente para que os elétrons sejam liberados do filamento e direcionados para o ânodo. A corrente do filamento é da ordem de amperes (A). A corrente do tubo é o fluxo de elétrons do cátodo para o ânodo. É controlado pela corrente de filamento. Uma pequena mudança na corrente do filamento resulta em uma mudança exponencial na corrente do tubo. A corrente do tubo é da ordem de miliamperes.
    \end{itemize}

	\textcolor{MediumOrchid}{\LobsterTwo\textbf{Aceleradores Lineares e Cobalto-60}}
	\begin{itemize}
		\item O filtro aplanador é usado em feixes de fótons para criar um perfil de dose plana a uma profundidade de 10 cm na água. Sem ele, o perfil do feixe é muito pontiagudo (mais intenso no centro). O filtro aplanador produz ``chifres'' nas bordas do feixe em profundidades menores. É constituído tipicamente por uma peça cônica de material de alto Z. Devido à sua introdução na direção do feixe, a energia efetiva do feixe é aumentada, devido a filtração dos fótons de baixa energia (endurecimento do feixe), porém diminui a intensidade do feixe e, portanto, a taxa de dose.
		
		\item Ao trocar do modo fóton para o modo elétron no linac, No modo elétron, o alvo e o filtro aplanador são removidos. Uma folha espalhadora é adicionada para criar um feixe amplo e plano de elétrons. Aplicadores de elétrons são utilizados para colimar o feixe e remover a dispersão no ar aproximando o feixe da superfície da pele. A corrente do feixe acelerado é muito menor no modo elétron, pois não há perda de eficiência como visto no modo fóton devido à presença do alvo e do filtro aplanador.
		
		\item Um canhão de elétrons (gun) é um acelerador eletrostático simples contendo um cátodo com um filamento aquecido e um ânodo aterrado perfurado. Um cátodo aquecido emite elétrons, que são acelerados em direção a um ânodo, passando por uma abertura para alcançar o guia de onda aceleradora. Ao longo desse caminho, eletrodos de foco carregados negativamente estreitam os elétrons em um feixe fino que passa pela abertura no ânodo. Uma grid permite a sincronização da liberação de elétrons no guia de onda de aceleração para corresponder à sua fase.
		
		\item O ímã de flexão (bending magnet) redireciona os elétrons acelerados em direção ao isocentro. O uso de apenas um ímã de flexão de 90° no projeto de um linac permite que o acelerador seja orientado paralelamente ao chão para que ele possa girar mais facilmente em torno de seu isocentro sem atingir o teto, o chão ou o paciente. Usando um ímã de curvatura de 270°, também permite que a energia dos elétrons seja selecionada, garantindo que as pequenas variações na energia do feixe estreito sejam curvadas com o mesmo raio e o feixe de elétrons seja focado, melhorando assim a penumbra do feixe.
		
		\item Aceleradores lineares modernos (linacs) são construídos de modo que o eixo de rotação do gantry, do colimador e da mesa sejam coincidentes em um ponto no espaço chamado isocentro. Isso permite que um paciente seja tratado com o isocentro dentro do alvo e que o linac gire ao redor do paciente.
		
		\item As câmaras monitoras em um linac monitoram o output do linac. Durante a calibração do Linac, a carga coletada nessas câmaras de ionização (representadas por unidades monitoras) é correlacionada à dose fornecida. As câmaras monitoras são então capazes de desligar o feixe uma vez que a dose desejada tenha sido entregue. Câmaras múltiplas ou divididas também são usadas para rastrear a planura e a simetria do feixe.
		
		\item Os guias de ondas de aceleração podem ser de ondas estacionárias ou ondas ``viajantes''. Os guias de ondas de ondas estacionárias são mais eficientes e mais curtos do que os guias de ondas viajantes equivalentes. Para criar uma onda estacionária, as micro-ondas são refletidas no final, em vez de sair do guia de ondas. Isso gera uma superposição das ondas em cada segunda cavidade, fornecendo o dobro da potência de aceleração de uma cavidade de onda progressiva. As cavidades intercaladas, onde as micro-ondas refletidas se cancelam, são deslocadas para o lado do caminho do elétron, pois não fornecem nenhuma aceleração. Assim, a potência de aceleração é aumentada e o guia de onda é encurtado.
		
		\item O \ce{SF_{6}} é um dielétrico e evita o arco dentro do guia de ondas de transmissão que conduz as micro-ondas a partir de sua fonte para a guia de ondas de aceleração.
		
		\item Um linac usa microondas a 3.000 MHz ou 3 GHZ. Estes são chamados de microondas de banda S. Alguns aceleradores mais compactos, como o CyberKnife, usam micro-ondas da banda X em torno de 10 GHz. O tamanho do guia de ondas depende do comprimento de onda do micro-ondas, portanto, uma frequência mais alta implica em um guia de ondas mais curto.
		
		\item Uma klystron amplifica as microondas de baixa potência em microondas de alta potência. À medida que os elétrons são enviados através de um ``tubo de deriva'' (drift tube), sua velocidade é modulada pelo campo elétrico alternado na frequência das micro-ondas de baixa potência entrantes, criando “montes” de elétrons. Os “montes” de elétrons induzem cargas no final da cavidade, criando microondas de maior potência na mesma frequência.
		
		\item Uma magnetron é um gerador de micro-ondas. Tem uma estrutura circular com um cátodo no centro e um ânodo na superfície externa composta por cavidades ressonantes. Os elétrons são produzidos no cátodo e são submetidos a um campo elétrico entre o ânodo e o cátodo. Um campo magnético estático é aplicado perpendicularmente ao campo elétrico e ao movimento dos elétrons. Os elétrons se movem em espirais em direção às cavidades, criando energia de micro-ondas, que é então enviada para a guia de onda acelerador.
		
		\item No acelerador linear, o feixe de elétrons sai da guia de onda de aceleração e então é curvado pelo ímã em direção ao alvo. O feixe de raios X do alvo é moldado pelo colimador primário e, em seguida, pelo filtro aplanador, medido pelas câmaras monitoras e, em seguida, moldado pelos colimadores secundários e colimadores multi-lâminas (MLCs).
		
		\item Uma Tomoterapia é um sistema integrado com um linac de 6 MV montado em um gantry em forma de anel semelhante a uma CT. Um MVCT pode ser adquirido na posição de tratamento antes do tratamento.
		
		\item O CyberKnife possui um acelerador linear de 6 MV (linac) montado em um braço robótico industrial. Ele pode posicionar o linac para direcionar feixes de tratamento de muitos ângulos não coplanares. Ele também possui orientação por imagem (kV estereoscópico e rastreamento de superfície infravermelho) que monitora continuamente a posição do paciente durante todo o processo de tratamento.
		
		\item O Mobetron é um acelerador linear de elétrons compacto portátil usado para radioterapia intraoperatória (IORT). É auto-blindado para que possa ser usado em qualquer sala de cirurgia. Suas energias de tratamento são 6, 9 ou 12 MeV. Por causa de sua curta fonte de 50 cm de distância à superfície (SSD), a taxa de dose é bastante alta (1.000 cGy/min). Isso leva a tempos de tratamento curtos.
		
		\item O Zeiss intrabeam é um aparelho utilizado para radioterapia intraoperatória. O canhão de elétrons injeta elétrons no acelerador com tensão máxima de 50 kV. Os elétrons são direcionados para um alvo de ouro que é colocado no centro de um aplicador esférico de acrílico. Raios X de aproximadamente 20 keV são emitidos isotropicamente.
		
		\item As principais vantagens de usar um feixe FFF são: A taxa de dose pode ser de três a cinco vezes maior, o que resulta em tratamentos mais rápidos. A ausência de um filtro aplanador reduz significativamente a quantidade de radiação espalhada. Isso pode reduzir a dose corporal total para o paciente e também facilitar a modelagem do feixe em um sistema de planejamento de tratamento.
		
		\item As principais desvantagens em usar um feixe FFF, o perfil do feixe apresenta um pico pano seu centro. O feixe destina-se apenas a tratar pequenos alvos. Alvos muito grandes exigiriam mais unidades monitoras para tratar o volume do tumor que está longe do eixo central. O feixe também é ligeiramente menos penetrante do que um feixe plano que é endurecido pela presença do filtro aplanador.
		
		\item Na utilização de filtros físicos, O ângulo do filtro é geralmente definido como o ângulo da curva de isodose de 50\% em relação ao eixo central a uma profundidade de referência de 10 cm.
		
		\item O fator de transmissão do filtro é definido como a razão entre a dose com a presença do filtro e a dose sem a presença do filtro a uma profundidade de 10 cm.
		
		\item O Filtro Físico é um bloco de aço montado na bandeja, colocado no slot de acessório de um acelerador linear (linac). O aço tem a forma de uma cunha, com uma extremidade mais grossa que a outra. A ponta mais grossa, conhecida como calcanhar (heel), modula mais o feixe do que a outra ponta, (toe). Os filtros físicos estão disponíveis em ângulos padrão fixos, geralmente 15°, 30°, 45° e 60°.
		
		\item O filtro universal é semelhante a um filtro físico de 60°, mas é integrado na cabeça do linac. O filtro se move para dentro e para fora do feixe de tratamento, dependendo do ângulo de isodose do filtro desejado. Um ângulo de isodose de um filtro de 60° exigiria que o filtro estivesse no feixe durante o campo de tratamento. Para um ângulo de isodose de filtro menor, o filtro se moveria para fora do campo de tratamento por uma fração do tratamento.
		
		\item o Filtro dinâmico cria uma isodose angulada movendo o jaw do colimador pelo campo. O ângulo da isodose do filtro pode ser alterado modulando a taxa de dose e/ou alterando a velocidade do jaw.
		.
		\item O fator de dose absorvida é inversamente proporcional ao quadrado da distância do alvo. Isso se deve à divergência do feixe. Imagine uma esfera ao redor do alvo com raio r. A área da superfície da esfera é $4\pi r^2$, com toda a radiação do alvo passando por ela. Uma esfera duas vezes mais distante do alvo, a mesma quantidade de radiação passará por sua superfície, mas a área da superfície é quatro vezes maior, reduzindo efetivamente a intensidade em qualquer ponto em quatro.
		
		\item Se um cálculo de tratamento é alterado de 10 cm de profundidade e 100 cm de distância da fonte à superfície (SSD) para 10 cm de profundidade da fonte à distância do eixo (SAD), o fator SSD é substituído por um fator SAD e o percentual de dose na produndidade (PDP) é substituída pela razão tecido máximo (TMR). As unidades monitoras diminuirão devido ao paciente estar 10 cm mais próximo da fonte.
		
        \item Em um MLC de foco duplo, a divergência do MLC corresponde à divergência do feixe de tratamento nas direções x e y. Em uma direção, as folhas se movem com uma direção arqueada para combinar as extremidades da lâmina com a divergência do feixe. Na outra direção, a divergência é igualada pela variação da largura da seção transversal das lâminas. As lâminas são mais finas no final, mais perto do alvo. Ambos resultam em uma penumbra mais nítida.

        \item A meia-vida do Co-60 é de 5,26 anos. Para fins práticos, o Co-60 é considerado inofensivo e inativo após 10 meias-vidas. Assim, o Co-60 precisaria ser armazenado com segurança por aproximadamente 53 anos.
        
        \item Um núcleo de cobalto-60 (Z = 27) é produzido em um reator nuclear bombardeando átomos estáveis de Co-59 com nêutrons. Para decair para um estado estável, o núcleo do Co-60 primeiro emite uma partícula $\beta^-$ (decaimento $\beta^-$) para o Níquel-60 (Z = 28) e, em seguida, são observadas duas emissões gama com energias de 1,17 e 1,33 MeV. A energia média dos gamas é de 1,25 MeV.
        
        \item Em um acelerador eletrostático, as partículas são aceleradas por um campo eletrostático constante devido a uma diferença de potencial aplicada. A energia cinética que a partícula pode ganhar é limitada pela diferença de voltagem máxima do campo eletrostático. Nos aceleradores cíclicos, os campos elétricos são variáveis e as partículas passam repetidamente por eles. Isso requer campos magnéticos ou raios de curvatura cada vez mais fortes para manter as partículas retornando à cavidade. Exemplos de aceleradores eletrostáticos usados na medicina são tubos de raios X superficiais e de ortovoltagem e geradores de nêutrons. Exemplos de aceleradores cíclicos são microtrons, betatrons, ciclotrons e síncrotrons.
        
        \item Exemplos de máquinas clínicas de megavoltagem são aceleradores como o gerador Van de Graaff, acelerador linear (linac), betatron, ciclotron e microtron, e unidades de raios gama de teleterapia, como cobalto-60.
        
        \item \textbf{Ciclotron}: Um acelerador circular onde partículas carregadas geradas em uma fonte central são aceleradas em espiral para fora em um plano perpendicular a um campo magnético fixo por um campo elétrico alternado. Um ciclotron é capaz de gerar energias de partículas entre 1 e 30 MeV.
        
        \item \textbf{Betatron}: Desenvolvido em 1940 para a aceleração circular induzida de elétrons e partículas leves. O guia do campo magnético é aumentado ao longo do tempo para manter as partículas em um círculo de diâmetro constante. A energia média é de 45 MeV (energia máxima ~300 MeV). Após o advento dos aceleradores lineares (linacs), eles não foram mais utilizados devido às suas grandes dimensões, altos custos e baixas taxas de dose.
        
        \item \textbf{Microtron}: Entrou nas clínicas em 1972 como uma combinação de um acelerador linear, seguido por ímãs para curvar os elétrons para reentrar no acelerador. Pode produzir energias de elétrons de até 50 MeV. Um mícroton é usado para fornecer elétrons para mais de uma sala de tratamento.
        
        \item A principal vantagem dos prótons de alta energia e outras partículas carregadas pesadas é sua distribuição de dose característica em função da profundidade. À medida que o feixe atravessa os tecidos, a dose depositada é aproximadamente constante com a profundidade até perto do final do seu alcance, onde a dose atinge o pico, seguida por uma rápida queda para zero. A região de alta dose no final do alcance das partículas é chamada de pico de Bragg.
        
        \item O alcance aproximado para outras partículas com a mesma velocidade inicial pode ser calculado pela seguinte relação:
        
            $$\frac{R_1}{R_2} = \left(\frac{M_1}{M_2}\right) \times \left(\frac{Z_2}{Z_1}\right)^2$$
        onde $R_1$ e $R_2$ são os alcances das partículas, $M_1$ e $M_2$ são suas massas e $Z_1$ e $Z_2$ são as cargas das duas partículas que estão sendo comparadas. Assim, a partir dos dados de energia de alcance para prótons, pode-se calcular o alcance de outras partículas, como partículas alfa e íons de carbono.

        \item Feixes de nêutrons de alta energia para radioterapia são produzidos bombardeando um alvo com partículas carregadas em um ciclotron ou acelerador linear ou em um gerador de deutério-trítio (D-T). Como os nêutrons não têm carga elétrica, eles não podem ser direcionados ou focados. Os nêutrons produzidos viajam principalmente na direção das partículas que chegam. As partículas para o bombardeio são deutérios (D = \ce{^{2}H}) ou prótons  que atingem um alvo geralmente feito de berílio, exceto no gerador D–T no qual o trítio (T = \ce{^{3}H}) é usado como alvo.
        
        \item O rádio decai para radônio por emissão alfa. Uma partícula alfa é ejetada com energia de 4,78 MeV ou 4,6 MeV. A partícula alfa de 4,6 MeV é acompanhada por um raio gama de 0,18 MeV.
        
        \item A PDP para feixes de elétrons apresenta uma dose de superfície alta, seguida por uma curva gradual até a profundidade da dose máxima (d\textsubscript{max}). A profundidade da dose máxima aumenta com o aumento da energia. Além  de d\textsubscript{max}, a dose de profundidade diminui muito rapidamente. R\textsubscript{50} é definido como a profundidade na qual a dose depositada é metade da dose máxima. o Alcance prático ou R\textsubscript{p}, é definido como o ponto de interseção no qual a curva de dose de profundidade é extrapolada até a cauda de bremsstrahlung.
        
    \end{itemize}
\end{exemplo}


\bibliography{ref.bib}
\end{document}