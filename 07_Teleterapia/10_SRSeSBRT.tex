\documentclass[11pt,a4paper]{article}
\usepackage[top=3cm, bottom=2cm, left=2cm, right=2cm]{geometry}
\usepackage[utf8]{inputenc}
\usepackage{amsmath, amsfonts, amssymb}
\usepackage{siunitx}
\usepackage[brazil]{babel}
\usepackage{graphicx}
\usepackage[margin=10pt,font={small, it},labelfont=bf, textfont=it]{caption}
\usepackage[dvipsnames, svgnames]{xcolor}
\DeclareCaptionFont{MediumOrchid}{\color[svgnames]{MediumOrchid}}
\usepackage[pdftex]{hyperref}
\usepackage{natbib}
\bibliographystyle{plainnat}
\bibpunct{\textcolor{MediumOrchid}{\textbf{[}}}{\textcolor{MediumOrchid}{\textbf{]}}}{,}{s}{}{}
\usepackage{color}
\usepackage{footnote}
\usepackage{setspace}
\usepackage{booktabs}
\usepackage{multirow}
\usepackage{subfigure}
\usepackage{fancyhdr}
\usepackage{leading}
\usepackage{indentfirst}
\usepackage{wrapfig}
\usepackage{mdframed}
\usepackage{etoolbox}
\usepackage[version=4]{mhchem}
\usepackage{enumitem}
\usepackage{caption}
\usepackage{titlesec}
\usepackage{tcolorbox}
\usepackage{tikz}
\usepackage{LobsterTwo}
\usepackage[T1]{fontenc}
\usepackage{fontspec}
\usepackage{txfonts}
\usepackage[bottom]{footmisc}

\makeatletter
\def\footnoterule{\kern-3pt\color{MediumOrchid}\hrule\@width0.6\textwidth height 0.8pt\kern2.6pt}
\makeatother

\renewcommand{\footnotelayout}{\itshape\color{MediumOrchid}}

\AtBeginEnvironment{equation}{\fontsize{13}{16}\selectfont}


\titleformat{\section}{\LobsterTwo\LARGE\color{CarnationPink}}{\thesection.}{1em}{}
\titleformat{\subsection}{\LobsterTwo\LARGE\color{CarnationPink}}{\thesubsection}{1em}{}


\DeclareCaptionLabelFormat{figuras}{\textcolor{DarkTurquoise}{Figura \arabic{figure}}}
\captionsetup[figure]{labelformat=figuras}

\makeatletter
\renewcommand\tagform@[1]{\maketag@@@{\color{CarnationPink}(#1)}}
\makeatother

\renewcommand{\theequation}{Eq. \arabic{equation}}
\renewcommand{\thefigure}{Fig. \arabic{figure}}
\renewcommand{\thesection}{\textcolor{CarnationPink}{\arabic{section}}}

\setlist[itemize]{label=\textcolor{CarnationPink}{$\blacksquare$}}

\setlist[enumerate]{label=\textcolor{CarnationPink}{\arabic*.}, align=left, leftmargin=1.5cm}


\newcounter{exemplo}

\NewDocumentEnvironment{exemplo}{ O{} }{%
\allowbreak
\setlength{\parindent}{0pt}
  \begin{mdframed}[
  leftline=true,
  topline=false,
  rightline=false,
  bottomline=false,
  linewidth=2pt,
  linecolor=CarnationPink,
  frametitlerule=false,
  frametitlefont=\LobsterTwo\large\color{CarnationPink},
  frametitle={\color{CarnationPink}\LobsterTwo\large #1},
  ]
}{%
  \end{mdframed}
}

\setlength{\fboxsep}{5pt}
\setlength{\fboxrule}{1.5pt}
\usepackage{float}
\renewcommand{\thefootnote}{\alph{footnote}}
\usepackage{url}
\hypersetup{
	colorlinks=true,
	linkcolor=DarkTurquoise,
	filecolor=DarkTurquoise,      
	urlcolor=DarkTurquoise,
	citecolor=DarkTurquoise,
	pdftitle={Especialista em Física da Radioterapia}
}
\pagestyle{fancy}
\fancyhf{}
\renewcommand{\headrulewidth}{0pt}
\rfoot{Página \thepage}

\title{\LobsterTwo\Huge{Radioterapia}}
\author{\LobsterTwo\Large{SRS e SBRT}\nocite{*}}
\date{\LobsterTwo\textit{Dalila Mendonça}}
\begin{document}
	\maketitle

\section{Introdução}

	A radiocirurgia estereotáxica (SRS) foi originalmente definida com o cumprimento de todas as seguintes condições:

	\begin{enumerate}[label=\textcolor{CarnationPink}{(\roman*)}]
		\item Tratamento com fração única;
		\item Alta dose por fração ($>$ 5Gy)
		\item Diâmetro do alvo $<$ 3.5 cm no cérebro;
		\item Precisão na entrega da dose $<$ 1 mm como definido pelo teste de Winston-Lutz\cite{lutz1988system};
		\item Nenhuma margem para PTV ou ITV é usada; podendo ser utilizada as margens para o CTV, mas no geral trata apenas o GTV; 
	\end{enumerate}

	A SRS foi posteriormente expandida para incluir também tratamentos de lesões na coluna vertebral com fração única e também para incluir tratamentos fracionados. A definição atual inclui:

	\begin{enumerate}[label=\textcolor{CarnationPink}{(\roman*)}]
		\item Tratamentos de 1 até 5 frações;
		\item Alta dose por fração (acima de 5 Gy);
		\item Diâmetro do alvo menor que 3.5 cm no sistema nervoso central (SNC  - crânio ou medula);
		\item Precisão na entrega da dose $<$ 1 mm como definido pelo resultado do teste end-to-end;
	\end{enumerate}

	A SBRT (Stereotactic Body Radiation Therapy), as vezes também chamada de SABRT (Stereotactic Ablative Radiation Therapy) é definida pelas seguintes condições:

	\begin{enumerate}[label=\textcolor{CarnationPink}{(\roman*)}]
		\item Tratamentos de 1 a 5 frações (Até 8 frações no Canadá);
		\item Altas doses por fração (acima de 5 Gy);
		\item Tamanho do alvo de até 4 cm de diâmetro para lesões no pulmão ou de 5 cm até 7 cm de diâmetro para alvos localizados nas cavidades torácicas e abdominais.
		\item Precisão na entrega do tratamento $<$ 1.5 mm até 2 mm definidas pelos resultados dos testes end-to-end;
		\item As margens para o ITV e para o PTV são utilizadas para compensar a movimentação intra-fração e inter-fração e deformações. A maioria dos alvos, exceto a próstata, necessitam de um gerenciamento do movimento respiratório.
	\end{enumerate}

\section{Requerimentos Técnicos}

	As altas doses e os acentuados gradientes de dose entregues em poucas frações nos tratamentos de SRS e SBRT resultam eum uma margem de erro muito menor que o os fracionamentos de radioterapia convencionais. Uma máquina utilizada para a tratamentos de SRS/SBRT deve, portanto, atender no mínimo aos seguintes requisitos técnicos mais rigorosos:

	\begin{enumerate}[label=\textcolor{CarnationPink}{(\roman*)}]
		\item A máquina deve atender às tolerâncias mecânicas descritas no TG-142;
		\item Os campos pequenos devem estar comissionados;
		\item Recursos de orientação pro imagem ( ou frame);
		\item Resultados do teste end-to-end (E2E) $<$ 1 mm e resultados do QA entrega (DQA) $<$3\%/2 mm.
		\item Técnicas de gerenciamento do movimento respiratório;
	\end{enumerate}

	O teste E2E é uma adaptação do teste Winston-Lutz para a era do SRS/SBRT frameless guiada por imagem. Idealmente, no teste E2E, cada uma das etapas deve ser realizada pelo(s) membro(s) da equipe envolvidos no tratamento que desempenharão esta mesma etapa no tratamento de um paciente. A realização de um teste E2E da maneira apresentada a seguir como parte do processo de comissionamento, pode ajudar a esclarecer o fluxo de tratamento e o fluxo de informações e também pode servir como uma ferramenta para desenvolver o conjunto inicial de políticas e procedimentos (POP):

	\begin{enumerate}[label=\LobsterTwo\textbf{\textcolor{CarnationPink}{\arabic*${}^\circ$}}]
		\item \LobsterTwo\textbf{\textcolor{CarnationPink}{$\star$ Físico Médico $\star$}} Um phantom  antropomórfico apropriado para o local de tratamento, equipado com fiduciais para localização semelhante ao que será usado em um paciente, é carregado com ferramentas de dosimetria in vivo, como dosímetro termoluminescente (TLD), detector luminescente opticamente estimulado (OSLD), metal- transistores de efeito de campo semicondutores de óxido (MOSFETs), gel ou filme (físico médico).
		
		\item \LobsterTwo\textbf{\textcolor{DarkTurquoise}{$\star$ Técnico $\star$}} O phantom é imobilizado e adquirida sua imagem, usando os mesmos dispositivos de imobilização e protocolos de aquisição que seriam usados para o paciente.
		
		\item \LobsterTwo\textbf{\textcolor{MediumOrchid}{$\star$ Médico $\star$}} As imagens são importadas para o sistema de planejamento de tratamento e o alvo oculto é contornado.
		
		\item \LobsterTwo\textbf{\textcolor{DarkTurquoise}{$\star$ Dosimetrista $\star$}}Um plano de tratamento é desenvolvido cumprindo as restrições de dose prescritas .
		
		\item \LobsterTwo\textbf{\textcolor{MediumOrchid}{$\star$ Médico $\star$}}O plano é assinado e revisado.
		
		\item \LobsterTwo\textbf{\textcolor{DarkTurquoise}{$\star$ Dosimetrista $\star$}} A documentação do plano é gerada e o plano é exportado para a unidade de tratamento.
		
		\item \LobsterTwo\textbf{\textcolor{CarnationPink}{$\star$ Físico Médico $\star$}} Uma segunda verificação, cálculo da unidade de monitor secundário (MU) e medições de controle de qualidade específicas do paciente, se aplicável, são realizadas.
		
		\item \LobsterTwo\textbf{\textcolor{DarkTurquoise}{$\star$ Técnico $\star$}}O tratamento é entregue .
		\item \LobsterTwo\textbf{\textcolor{CarnationPink}{$\star$ Físico Médico $\star$}} O resultado da medição da dose é analisado.
	\end{enumerate}


\section{Políticas e Procedimentos}

	Antes de iniciar um programa de SRS/SBRT, uma série de políticas e procedimentos P\&Ps devem ser desenvolvidas uma configuração multidisciplinar de todos os prestadores de cuidados envolvidos. Para determinar quais P\&Ps são necessários, um fluxograma do paciente pode servir como uma ferramenta útil. Este fluxograma também servirá como uma ferramenta de controle de qualidade para garantir que todos os membros da equipe de tratamento do paciente entendam seu papel no processo de atendimento, a ordem das etapas do atendimento e o fluxo de informações.

	A Tabela 17.1 descreve um exemplo de um conjunto de P&Ps para atender às diretrizes de credenciamento da prática do American College of Radiology (ACR) e ao escopo das diretrizes da prática da Canadian Association of Radiation Oncologists (CARO) para recomendações SBRT de pulmão, fígado e coluna. O white paper da ASTRO sobre considerações de qualidade e segurança no SBRT enfatiza que o SRS/SBRT não é uma técnica/modalidade e que a especialização em uma área-alvo anatômica não constitui especialização em outros locais. Portanto, é necessário desenvolver procedimentos específicos para a tecnologia em uso na instituição, bem como específicos para o(s) local(is) da doença. 

	Os P\&Ps devem ser testados usando um cenário simulado de tratamento do paciente, preferencialmente incluindo a aquisição de imagens, planejamento de tratamento e entrega de tratamento de um plano de cuidados do paciente a um phantom. Após o tratamento dos pacientes iniciais, as P\&Ps devem ser revisadas. Após a revisão inicial, os P\&Ps devem ser revisados e, se aplicável, atualizados anualmente.

\section{Seleção do Paciente e Desenvolvimento de Protocolos de Tratamento}



\section{Imagem de Simulação}


\section{Margens}


\section{Constraints de Dose e Planejamento do Tratamento}

\subsection*{RTOG e Outros Protocolos}

\section{Entrega do Tratamento}



\bibliography{ref.bib}
\end{document}