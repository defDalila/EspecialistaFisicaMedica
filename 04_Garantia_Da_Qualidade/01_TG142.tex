\documentclass[11pt,a4paper]{article}
\usepackage[top=3cm, bottom=2cm, left=3cm, right=2cm]{geometry}
\usepackage[utf8]{inputenc}
% \usepackage[T1]{fontenc}
\usepackage{amsmath, amsfonts, amssymb}
\usepackage{siunitx}
\usepackage[brazil]{babel}
\usepackage{graphicx}
\usepackage[margin=10pt,font={small, it},labelfont=bf, textfont=it]{caption}
\usepackage[dvipsnames, svgnames]{xcolor}
\DeclareCaptionFont{MediumOrchid}{\color[svgnames]{MediumOrchid}}
\usepackage[pdftex]{hyperref}
\usepackage{natbib}
\bibliographystyle{plainnat}
\bibpunct{[}{]}{,}{s}{}{}
\usepackage{color}
\usepackage{footnote}
\usepackage{setspace}
\usepackage{booktabs}
\usepackage{multirow}
\usepackage{subfigure}
\usepackage{fancyhdr}
\usepackage{leading}
\usepackage{indentfirst}
\usepackage{wrapfig}
\usepackage{mdframed}
\usepackage{etoolbox}
\usepackage[version=4]{mhchem}
\usepackage{enumitem}
\usepackage{caption}
\usepackage{titlesec}




\titleformat{\section}{\LARGE\color{CarnationPink}}{\thesection}{1em}{}
\titleformat{\subsection}{\LARGE\color{CarnationPink}}{\thesubsection}{1em}{}


\DeclareCaptionLabelFormat{figuras}{\textcolor{CarnationPink}{Figura \arabic{figure}}}
\captionsetup[figure]{labelformat=figuras}

\makeatletter
\renewcommand\tagform@[1]{\maketag@@@{\color{CarnationPink}(#1)}}
\makeatother

\renewcommand{\theequation}{Eq. \arabic{equation}}
\renewcommand{\thefigure}{Fig. \arabic{figure}}
\renewcommand{\thesection}{\textcolor{CarnationPink}{\arabic{section}}}

\setlist[itemize]{label=\textcolor{CarnationPink}{$\mathbf{\square}$}}

\setlist[enumerate]{label=\textcolor{CarnationPink}{\arabic*.}, align=left}


\newcounter{exemplo}

\NewDocumentEnvironment{exemplo}{ O{} }{%
\allowbreak
\setlength{\parindent}{0pt}
  \begin{mdframed}[
  leftline=true,
  topline=false,
  rightline=false,
  bottomline=false,
  linewidth=2pt,
  linecolor=CarnationPink,
  frametitlerule=false,
  frametitlefont=\Large\bfseries\color{CarnationPink},
  frametitle={\color{CarnationPink}\normalfont\bfseries #1},
  ]
}{%
  \end{mdframed}
}

\setlength{\fboxsep}{5pt}
\setlength{\fboxrule}{1.5pt}
\usepackage{float}
\renewcommand{\thefootnote}{\alph{footnote}}
\usepackage{url}
\hypersetup{
	colorlinks=true,
	linkcolor=DarkTurquoise,
	filecolor=DarkTurquoise,      
	urlcolor=DarkTurquoise,
	citecolor=DarkTurquoise,
	pdftitle={Radioterapia}
}
\pagestyle{fancy}
\fancyhf{}
\renewcommand{\headrulewidth}{0pt}
\rfoot{Página \thepage}

\title{Garantia da Qualidade}
\author{TG 142\nocite{*}}
\date{\textit{Dalila Mendonça}}
\begin{document}
	\maketitle


  \section{Objetivo}

    O objetivo do TG 142 foi de suprir as demandas de novas tecnologias que até então não eram contempladas pelo TG 40,sendo elas: MLC, Colimadores Assimétricos, Filtros dinâmicos, Filtros Virtuais, EPID, imagens KV estáticas, CBCT, gating respiratório e as técnicas de IMRT, SRS e SBRT.

    O Objetivo de um programa de QA é que seja mantido as características da máquina conforme adquiridos em seu comissionamento e aceite e , embora os testes possam ser realizados por diferentes profissionais, a responsabilidade dos testes é do Físico Medico Especialista.


  \section{Periodicidade dos Testes}

    \subsection{Testes Diários}

        \begin{table}[h]
            \centering
            \caption{Testes Diários}
            \begin{tabular}{p{8cm} p{2.5cm} p{1.5cm} p{2cm}}
                \toprule
                \multicolumn{4}{c}{\textbf{\textcolor{CarnationPink}{Testes Dosimétricos}}} \\
                \addlinespace[5pt]
                Procedimento & Não-IMRT & IMRT & SRS/SBRT \\
                \hline
                \addlinespace[5pt]
                Constância do output de raios - x e 
                constância do output dos feixes de elétrons
                \textit{\textbf{\small{\textcolor{DarkOrchid}{(Semanalmente, 
                exceto para máquinas com monitoramento único)}}}} & \multicolumn{3}{c}{\multirow{4}{*}{\textbf{\textcolor{cyan}{3\%}}}} \\
                \hline
                \hline
                \addlinespace[5pt]
                \multicolumn{4}{c}{\textbf{\textcolor{CarnationPink}{Testes Mecânicos}}} \\
                \addlinespace[5pt]
                Procedimento & Não-IMRT & IMRT & SRS/SBRT \\
                \hline
                \addlinespace[5pt]
                Localização do Laser & \textcolor{cyan}{\qty{2}{mm}} & \textcolor{cyan}{\qty{1.5}{mm}} & \textcolor{cyan}{\qty{1}{mm}} \\
                \addlinespace[5pt]
                Indicador de SSD óptico no iso
                \textit{\textbf{\small{\textcolor{DarkOrchid}{(Telémetro)}}}} & \multicolumn{3}{c}{\multirow{1}{*}{\textbf{\textcolor{cyan}{\qty{2}{mm}}}}} \\
                \addlinespace[5pt]
                Indicador de Tamanho do Colimador & \textcolor{cyan}{\qty{2}{mm}} & \textcolor{cyan}{\qty{2}{mm}} & \textcolor{cyan}{\qty{1}{mm}} \\




            \end{tabular}
        \end{table}

    \subsection{Testes Mensais}

    \subsection{Testes Anuais}

\bibliography{ref.bib}
\end{document}
