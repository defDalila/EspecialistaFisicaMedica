\documentclass[11pt,a4paper]{article}
\usepackage[top=3cm, bottom=2cm, left=3cm, right=2cm]{geometry}
\usepackage[utf8]{inputenc}
% \usepackage[T1]{fontenc}
\usepackage{amsmath, amsfonts, amssymb}
\usepackage{siunitx}
\usepackage[brazil]{babel}
\usepackage{graphicx}
\usepackage[margin=10pt,font={small, it},labelfont=bf, textfont=it]{caption}
\usepackage[dvipsnames, svgnames]{xcolor}
\DeclareCaptionFont{MediumOrchid}{\color[svgnames]{MediumOrchid}}
\usepackage[pdftex]{hyperref}
\usepackage{natbib}
\bibliographystyle{plainnat}
\bibpunct{[}{]}{,}{s}{}{}
\usepackage{color}
\usepackage{footnote}
\usepackage{setspace}
\usepackage{booktabs}
\usepackage{multirow}
\usepackage{subfigure}
\usepackage{fancyhdr}
\usepackage{leading}
\usepackage{indentfirst}
\usepackage{wrapfig}
\usepackage{mdframed}
\usepackage{etoolbox}
\usepackage[version=4]{mhchem}

\newcounter{exemplo}

\NewDocumentEnvironment{exemplo}{ O{} }{%
\allowbreak
\setlength{\parindent}{0pt}
  \begin{mdframed}[
  leftline=true,
  topline=false,
  rightline=false,
  bottomline=false,
  linewidth=2pt,
  linecolor=CarnationPink,
  frametitlerule=false,
  frametitlefont=\Large\bfseries\color{CarnationPink},
  frametitle={\color{CarnationPink}\normalfont\bfseries Exemplo: #1},
  ]
}{%
  \end{mdframed}
}

\setlength{\fboxsep}{10pt}
\setlength{\fboxrule}{1pt}
\usepackage{float}
\renewcommand{\thefootnote}{\alph{footnote}}
\usepackage{url}
\hypersetup{
    colorlinks=true,
    linkcolor=cyan,
    filecolor=cyan,      
    urlcolor=cyan,
    citecolor=cyan,
    pdftitle={Proteção Radiológica}
}
\pagestyle{fancy}
\fancyhf{}
\renewcommand{\headrulewidth}{0pt}
\rfoot{Página \thepage}
\title{Física Médica}
\author{Grandezas E Quantidades Dosimétricas\nocite{*}}
\date{\textit{Dalila Mendonça}}
\begin{document}
	\maketitle

    As grandezas relacionadas a Dosimetria das Radiações auxiliam na determinação quantitativa da energia depositada pela radiação no meio. 
    
    
    \section{Fluência e Fluência de Energia dos Fótons}

        Embora seja definida para fótons, a fluência pode ser utilizada por partículas carregadas. 

        \

        A \textit{\textbf{\textcolor{CarnationPink}{Fluência de Partículas $\mathbf{\Phi}$}}} é dado pela taxa de variação do número de partículas $dN$ incidentes em uma esfera com área de seção transversal $dA$, ou seja:

			\begin{equation}
			\Phi  = \frac{d N}{d A} 
			\end{equation}

    	\noindent A unidade de medida da fluência da partículas	$\Phi$ é \unit{m^{-2}}. O fato de considerar apenas uma esfera com seção transversal $dA$ está relacionado ao fato de que só é considerado na fluência uma área perpendicular a direção de propagação do feixe (uma vez que independente da angulação que o feixe está incidindo, sempre atravessará a mesma área transversal de uma esfera) e portanto não há dependência angular na fluência. Ao considerar uma fluência planar, avalia-se o número de fótons que atravessam um plano por unidade de área, e portanto a fluência irá depender do ângulo de incidência do feixe. 

      	\

      	A \textit{\textbf{\textcolor{CarnationPink}{Fluência de Energia $\mathbf{\Psi}$}}} E a razão entre a Energia Radiante Incidente ($dE$) e a área transversal de uma esfera ($dA$), ou seja:

			\begin{equation}
			\Psi = \frac{d E}{d A}
			\end{equation}

      	\noindent A unidade de medida da Fluência de Energia é \unit{J \cdot m^{-2}}. A Fluência de energia pode ser calculada através da Fluência de partículas através da relação:

			\begin{equation}
			\Psi = \frac{d N}{d A} \cdot E = \Phi \cdot E
			\end{equation}
      
      	\noindent onde $E$ é a energia da partícula e $\Phi$ é o número de partículas com energia E, caracterizada por um feixe monoenergético. 

    	\

    	A maior parte dos feixes de fótons e de outras partículas são polienergéticos, e portanto para estes feixes são utilizados o \textit{\textbf{\textcolor{CarnationPink}{Espectro de Fluência de Partículas $\Phi_E(E)$ e o Espectro de Fluência de Energia $\Psi_E(E)$}}}, onde:

			\begin{equation}
			\Phi_E(E) = \frac{d \Phi}{dE}(E)
			\end{equation}

      	\noindent e

			\begin{equation}
			\Psi_E(E) = \frac{d \Psi}{d E}(E) = \frac{d \Phi}{dE}(E) \cdot E 
			\end{equation}

		\

		A \textit{\textbf{\textcolor{CarnationPink}{Taxa De Fluência Da Partícula}}} é dada pela taxa de variação da fluência em relação ao tempo, ou seja:

			\begin{equation}
				\dot{\Phi} = \frac{d \Phi}{d t}
			\end{equation}

		\noindent no qual é dada em \unit{m^-2 \cdot s^-1}.

		\

		A \textit{\textbf{\textcolor{CarnationPink}{Taxa de Fluência de Energia}}}, também chamada de intensidade, é dada pela taxa de variação da fluência de energia em função do tempo, ou seja:

			\begin{equation}
				\dot{\Psi} = \frac{d \Psi}{d t}
			\end{equation}

		\noindent onde sua unidade de medida é \unit{W \cdot m^{-2}} ou \unit{J \cdot m^{-2} \cdot s^{-1}}.
		
	\section{Exposição}

		A Exposição ($X$) é a habilidade dos fótons ionizar o ar, dada pela seguinte equação:

			\begin{equation}
				X = \frac{\Delta Q}{\Delta m_{ar}}
			\end{equation}

		\noindent onde $\Delta Q$ é o valor absoluto da carga total dos íons de um sinal produzidos no ar, quando todos os elétrons e pósitrons liberados ou criados por fótons em  massa de ar $\Delta m_{ar}$ são completamente parados no ar. A unidade de medida no SI é \unit{C \cdot Kg^{-1}}; A unidade antiga da Exposição é o Roentgen (R) onde:

			\begin{equation*}
				1 R = 2.58 \times 10 ^{-4}\frac{C}{Kg}
			\end{equation*}

		\noindent porém está unidade já não é tão utilizada e portanto a unidade para Exposição é de \qty{2.58e{-4}}{C/Kg} de ar.

	\section{Kerma}

		KERMA é o acrônimo para Energia Cinética Liberada Por Unidade de Massa. É uma quantidade não estocástica aplicável às radiações indiretamente ionizantes, como prótons e nêutrons. Esta grandeza quantifica a quantidade média de energia transferida pela radiação indiretamente ionizante para radiações diretamente ionizantes, sem se preocupar com o que acontece após esta transferência de energia. 

		Considerando os fótons, o processo de ionização se dá por duas etapas: 

			\begin{enumerate}
				\item Os fótons transferem sua energia para as partículas carregadas secundárias, como os elétrons, por meio de diferentes processos de interação (Efeito Fotoelétrico, Efeito Compton, Produção de pares, etc \dots).
				
				\item A partícula secundária liberara irá então transferir sua energia através de processos como excitação e ionização. 
			\end{enumerate}

		O KERMA é então definido como a energia média transferida $d\bar{E}_{tr}$ de uma partícula indiretamente ionizante para partículas carregadas em um meio de massa $dm$, ou seja:

			\begin{equation}
				K = \frac{d \bar{E}_{tr}}{d m}
			\end{equation}

		\noindent onde a unidade para o KERMA é de \unit{J \cdot Kg^{-1}}


	\section{Dose Absorvida}

		A Dose absorvida é uma grandeza não-estocástica aplicável tanto para radiações indiretamente ionizantes quanto para radiações diretamente ionizantes. Para a radiação indiretamente ionizante, parte da sua energia é transferida como energia cinética para as partículas carregadas do meio, resultando no KERMA, e na sequência, as partículas carregadas perdem parcelas de sua energia para o meio, resultando na dose absorvida, e perdem parte da sua energia através da produção radiativa, em processos de bremmstrahlung ou aniquilação em voo. 
		
		
		A Dose  Absorvida está relacionada à uma quantidade estocástica chamada de energia transmitida ($\varepsilon$). A dose absorvida é definida como a energia média transmitida ($d\bar{\varepsilon}$) pela radiação diretamente ionizante para a matéria de massa $dm$ em um volume finito $V$, ou seja:


			\begin{equation}
				D = \frac{d \bar{\varepsilon}}{d m}
			\end{equation}

		\noindent A unidade para a Dose Absorvida é o Gy = \unit{J \cdot Kg ^{-1}}.

		A energia transmitida $\varepsilon$ é a soma de todas as energias entrando no volume de interesse subtraída da soma de todas as energias que saem do volume de interesse, considerando qualquer conversão de massa-energia dentro do volume. Por exemplo, a produção de pares diminui a energia em 1.022 MeV enquanto que a Aniquilação dos pares elétron-pósitron aumentam a energia pelo mesmo fator. 

		\textcolor{CarnationPink}{OBS:} Devido os elétrons viajarem no meio transferindo energia ao longo de sua trajetória, o local de deposição de energia é diferente do local onde a energia foi transferida pelo KERMA. 








  \bibliography{ref.bib}
\end{document}  