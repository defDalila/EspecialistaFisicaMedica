\documentclass[11pt,a4paper]{article}
\usepackage[top=3cm, bottom=2cm, left=2cm, right=2cm]{geometry}
\usepackage[utf8]{inputenc}
\usepackage{amsmath, amsfonts, amssymb}
\usepackage{siunitx}
\usepackage[brazil]{babel}
\usepackage{graphicx}
\usepackage[margin=10pt,font={small, it},labelfont=bf, textfont=it]{caption}
\usepackage[dvipsnames, svgnames]{xcolor}
\DeclareCaptionFont{MediumOrchid}{\color[svgnames]{MediumOrchid}}
\usepackage[pdftex]{hyperref}
\usepackage{natbib}
\bibliographystyle{plainnat}
\bibpunct{\textcolor{MediumOrchid}{\textbf{[}}}{\textcolor{MediumOrchid}{\textbf{]}}}{,}{s}{}{}
\usepackage{color}
\usepackage{footnote}
\usepackage{setspace}
\usepackage{booktabs}
\usepackage{multirow}
\usepackage{subfigure}
\usepackage{fancyhdr}
\usepackage{leading}
\usepackage{indentfirst}
\usepackage{wrapfig}
\usepackage{mdframed}
\usepackage{etoolbox}
\usepackage[version=4]{mhchem}
\usepackage{enumitem}
\usepackage{caption}
\usepackage{titlesec}
\usepackage{tcolorbox}
\usepackage{tikz}
\usepackage{LobsterTwo}
\usepackage[T1]{fontenc}
\usepackage{fontspec}
\usepackage{txfonts}
\usepackage[bottom]{footmisc}
\tcbuselibrary{skins,breakable}
\sisetup{output-decimal-marker={.}}

\makeatletter
\def\footnoterule{\kern-3pt\color{MediumOrchid}\hrule\@width0.6\textwidth height 0.8pt\kern2.6pt}
\makeatother

\renewcommand{\footnotelayout}{\itshape\color{MediumOrchid}}

\AtBeginEnvironment{equation}{\fontsize{13}{16}\selectfont}


\titleformat{\section}{\LobsterTwo\LARGE\color{CarnationPink}}{\thesection.}{1em}{}
\titleformat{\subsection}{\LobsterTwo\LARGE\color{CarnationPink}}{\thesubsection}{1em}{}
\titleformat{\subsubsection}{\LobsterTwo\large\color{CarnationPink}}{\thesubsubsection}{1em}{}


\DeclareCaptionLabelFormat{figuras}{\textcolor{DarkTurquoise}{Figura \arabic{figure}}}
\captionsetup[figure]{labelformat=figuras}

\makeatletter
\renewcommand\tagform@[1]{\maketag@@@{\color{CarnationPink}(#1)}}
\makeatother

\renewcommand{\theequation}{Eq. \arabic{equation}}
\renewcommand{\thefigure}{Fig. \arabic{figure}}
\renewcommand{\thesection}{\textcolor{CarnationPink}{\arabic{section}}}

\setlist[itemize]{label=\textcolor{CarnationPink}{$\blacksquare$}}

\setlist[enumerate]{label=\textcolor{CarnationPink}{\arabic*.}, align=left, leftmargin=1.5cm}


\newcounter{exemplo}

\NewDocumentEnvironment{exemplo}{ O{} }{%
\allowbreak
\setlength{\parindent}{0pt}
  \begin{mdframed}[
  leftline=true,
  topline=false,
  rightline=false,
  bottomline=false,
  linewidth=2pt,
  linecolor=CarnationPink,
  frametitlerule=false,
  frametitlefont=\LobsterTwo\large\color{CarnationPink},
  frametitle={\color{CarnationPink}\LobsterTwo\large #1},
  ]
}{%
  \end{mdframed}
}

\setlength{\fboxsep}{5pt}
\setlength{\fboxrule}{1.5pt}
\usepackage{float}
\renewcommand{\thefootnote}{\alph{footnote}}
\usepackage{url}
\hypersetup{
	colorlinks=true,
	linkcolor=DarkTurquoise,
	filecolor=DarkTurquoise,      
	urlcolor=DarkTurquoise,
	citecolor=DarkTurquoise,
	pdftitle={Especialista em Física da Radioterapia}
}
\pagestyle{fancy}
\fancyhf{}
\renewcommand{\headrulewidth}{0pt}
\rfoot{Página \thepage}

\title{\LobsterTwo\Huge{Radiobiologia}}
\author{\LobsterTwo\Large{Mecanismos de Interação da Radiação com o Tecido Biológico}\nocite{*}}
\date{\LobsterTwo\textit{Dalila Mendonça}}
\begin{document}
	\maketitle

\section{Introdução}

    Os raios X, raios gama e nêutrons são exemplos de radiação ionizante que podem causar danos biológicos por meio de uma combinação de ação direta e indireta. A ação direta ocorre quando a própria radiação interage diretamente com a molécula-alvo, resultando em ionização e excitação das moléculas biológicas. Isso pode levar a danos diretos no DNA, bem como em outras macromoléculas celulares.

    Por outro lado, a ação indireta ocorre quando a radiação interage com moléculas de água presentes no organismo. Essa interação resulta na produção de elétrons de recuo rápido, que podem ionizar outras moléculas, incluindo o DNA. Além disso, a radiólise da água gera radicais livres, como o radical hidroxila (\ce{OH^.}), que são altamente reativos e podem difundir-se por curtas distâncias para alcançar alvos críticos dentro das células. Estima-se que cerca de dois terços dos danos biológicos causados pelos raios X sejam resultado dessa ação indireta, envolvendo a formação de radicais livres.

    A utilização de protetores químicos é uma estratégia para modificar os danos biológicos causados pela radiação. Esses protetores químicos podem atuar de diferentes maneiras, como capturando radicais livres e impedindo sua interação com as moléculas celulares, ou estimulando sistemas de reparo do DNA para minimizar o dano. No entanto, é importante ressaltar que os protetores químicos podem ter uma eficácia limitada em relação a radiações de alta LET, como partículas α, devido à predominância do efeito direto dessas radiações.

    O processo físico de absorção da radiação ocorre muito rapidamente, em uma escala de tempo da ordem de picossegundos (\SI{e-15}{\second}). Após a absorção da energia da radiação, ocorrem uma série de reações químicas subsequentes que podem levar a alterações nas moléculas biológicas. Por exemplo, a vida útil dos radicais de DNA formados durante a interação da radiação com a molécula-alvo é da ordem de dezenas de microssegundos (\SI{e-5}{\second}). Durante esse tempo, esses radicais podem reagir com outras moléculas e causar danos adicionais.

    Os efeitos biológicos da radiação não ocorrem imediatamente após a exposição, mas podem se manifestar ao longo de diferentes períodos de tempo. A morte celular pode ocorrer horas ou dias após a exposição, quando as células danificadas tentam se dividir e não conseguem se recuperar. A carcinogênese, o desenvolvimento de câncer induzido pela radiação, pode ocorrer ao longo de meses, anos ou até mesmo décadas, dependendo da dose e do tipo de radiação. Além disso, os efeitos hereditários da radiação, que afetam as células germinativas e podem ser transmitidos para as gerações futuras, podem levar várias gerações para se tornarem aparentes.


\section{Efeitos Diretos e Indiretos}

    Os efeitos biológicos da radiação são principalmente devido ao dano causado ao DNA, que é considerado o alvo crítico. Independentemente do tipo de radiação absorvida pelos tecidos biológicos, como raios X, raios gama ou partículas carregadas, há potencial para interação direta com as estruturas celulares vitais. A ação direta ocorre quando a radiação interage diretamente com os alvos celulares, como o DNA, resultando em danos imediatos. Esse efeito direto da radiação é particularmente proeminente em radiações com alta transferência linear de energia (LET), como nêutrons ou partículas alfa, que depositam uma grande quantidade de energia em um curto espaço.

    Por outro lado, a ação indireta ocorre quando elétrons secundários gerados pela radiação interagem com outros átomos ou moléculas na célula. A molécula de água é particularmente importante nesse contexto, uma vez que constitui cerca de 80\% das células. Quando a radiação interage com uma molécula de água, ela pode ionizá-la, produzindo um íon radical \ce{H2O+} e um elétron livre ($e-$). O íon radical H2O+ é tanto um íon quanto um radical livre, pois possui um elétron desemparelhado. Esses íons radicais primários têm um tempo de vida muito curto e decaem para formar radicais livres, que não possuem carga, mas ainda possuem um elétron desemparelhado.

    O radical hidroxila (\ce{OH^{.}}) é um dos radicais livres produzidos pela interação da radiação com a água. Ele é altamente reativo devido à presença do elétron desemparelhado e pode difundir-se por uma curta distância na célula. Estima-se que cerca de dois terços dos danos causados pelos raios X no DNA de células mamíferas sejam atribuídos ao radical hidroxila. Esse radical pode reagir com o DNA e outros componentes celulares, causando danos. A ação indireta da radiação é influenciada por fatores químicos, como a presença de protetores ou sensibilizadores, que podem modular a formação e o efeito dos radicais livres.

    Quando consideramos a ação indireta da radiação, podemos resumir a sequência de eventos que leva às alterações biológicas observadas. Tudo começa com o incidente de um fóton de raios X, por exemplo. Esse fóton pode liberar elétrons rápidos ($e-$) durante a interação com os átomos do meio biológico. Esses elétrons rápidos podem então levar à formação de íons radicais, que são espécies químicas com carga elétrica. Os íons radicais, por sua vez, podem gerar radicais livres, que são moléculas ou átomos com elétrons desemparelhados. Esses radicais livres podem então causar mudanças químicas no ambiente celular, incluindo a quebra de ligações moleculares, que são eventos críticos para o surgimento de danos biológicos.

    A duração desses eventos ocorre em diferentes escalas de tempo. A ionização inicial pode ocorrer em picossegundos ($10^{-15}$ segundos), que é um tempo incrivelmente curto. Os radicais primários produzidos pela ejeção de elétrons têm uma vida útil de aproximadamente $10^{-10}$ segundos. O radical hidroxila \ce{OH^{.}} tem uma vida útil de cerca de $10^{-9}$ segundos dentro das células. Já os radicais de DNA formados por ionização direta ou reação com o radical hidroxila têm uma vida útil de cerca de $10^{-5}$ segundos na presença de ar. É importante ressaltar que o tempo entre a quebra de ligações e a expressão dos efeitos biológicos pode variar amplamente. Alguns efeitos, como a morte celular, podem ser observados em questão de horas ou dias, quando as células danificadas tentam se dividir. Por outro lado, a expressão de câncer induzido pela radiação pode ser retardada e se manifestar até 40 anos após a exposição. Mutações em células germinativas que levam a alterações hereditárias podem não ser aparentes por muitas gerações.


\bibliography{ref.bib}
\end{document}