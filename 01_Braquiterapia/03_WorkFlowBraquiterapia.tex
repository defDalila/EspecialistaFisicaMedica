\documentclass[11pt,a4paper]{article}
\usepackage[top=3cm, bottom=2cm, left=2cm, right=2cm]{geometry}
\usepackage[utf8]{inputenc}
\usepackage{amsmath, amsfonts, amssymb}
\usepackage{siunitx}
\usepackage[brazil]{babel}
\usepackage{graphicx}
\usepackage[margin=10pt,font={small, it},labelfont=bf, textfont=it]{caption}
\usepackage[dvipsnames, svgnames]{xcolor}
\DeclareCaptionFont{MediumOrchid}{\color[svgnames]{MediumOrchid}}
\usepackage[pdftex]{hyperref}
\usepackage{natbib}
\bibliographystyle{plainnat}
\bibpunct{[}{]}{,}{s}{}{}
\usepackage{color}
\usepackage{footnote}
\usepackage{setspace}
\usepackage{booktabs}
\usepackage{multirow}
\usepackage{subfigure}
\usepackage{fancyhdr}
\usepackage{leading}
\usepackage{indentfirst}
\usepackage{wrapfig}
\usepackage{mdframed}
\usepackage{etoolbox}
\usepackage[version=4]{mhchem}
\usepackage{enumitem}
\usepackage{caption}
\usepackage{titlesec}
\usepackage{tcolorbox}
\usepackage{tikz}
\usepackage{LobsterTwo}
\usepackage[T1]{fontenc}
\usepackage{fontspec}
\usepackage{txfonts}
\AtBeginEnvironment{equation}{\fontsize{13}{16}\selectfont}


\titleformat{\section}{\LobsterTwo\LARGE\color{CarnationPink}}{\thesection.}{1em}{}
\titleformat{\subsection}{\LobsterTwo\LARGE\color{CarnationPink}}{\thesubsection}{1em}{}


\DeclareCaptionLabelFormat{figuras}{\textcolor{DarkTurquoise}{Figura \arabic{figure}}}
\captionsetup[figure]{labelformat=figuras}

\makeatletter
\renewcommand\tagform@[1]{\maketag@@@{\color{CarnationPink}(#1)}}
\makeatother

\renewcommand{\theequation}{Eq. \arabic{equation}}
\renewcommand{\thefigure}{Fig. \arabic{figure}}
\renewcommand{\thesection}{\textcolor{CarnationPink}{\arabic{section}}}

\setlist[itemize]{label=\textcolor{CarnationPink}{$\mathbf{\square}$}}

\setlist[enumerate]{label=\textcolor{CarnationPink}{\arabic*.}, align=left}


\newcounter{exemplo}

\NewDocumentEnvironment{exemplo}{ O{} }{%
\allowbreak
\setlength{\parindent}{0pt}
  \begin{mdframed}[
  leftline=true,
  topline=false,
  rightline=false,
  bottomline=false,
  linewidth=2pt,
  linecolor=CarnationPink,
  frametitlerule=false,
  frametitlefont=\LobsterTwo\large\color{CarnationPink},
  frametitle={\color{CarnationPink}\LobsterTwo\large #1},
  ]
}{%
  \end{mdframed}
}

\setlength{\fboxsep}{5pt}
\setlength{\fboxrule}{1.5pt}
\usepackage{float}
\renewcommand{\thefootnote}{\alph{footnote}}
\usepackage{url}
\hypersetup{
	colorlinks=true,
	linkcolor=DarkTurquoise,
	filecolor=DarkTurquoise,      
	urlcolor=DarkTurquoise,
	citecolor=DarkTurquoise,
	pdftitle={Especialista em Física da Radioterapia}
}
\pagestyle{fancy}
\fancyhf{}
\renewcommand{\headrulewidth}{0pt}
\rfoot{Página \thepage}

\title{\LobsterTwo\Huge{Braquiterapia}}
\author{\LobsterTwo\Large{Workflow em Braquiterapia}\nocite{*}}
\date{\LobsterTwo\textit{Dalila Mendonça}}
\begin{document}
	\maketitle

\section{Introdução}

	A braquiterapia é definida como a aplicação temporária ou permanente de pequenas fontes radioativas seladas próximas ou dentro do volume alvo. A distribuição da dose de tratamento é caracterizada por alta dose localizada e queda abrupta da dose. Logo depois que o rádio foi isolado quimicamente por Marie e Pierre Curie, os efeitos dos danos causados pela radiação na pele foram observados e levaram à primeira aplicação de material radioativo para o tratamento de tumores superficiais. 

	Algumas décadas depois, as agulhas de rádio foram usadas em implantes intersticiais de baixa dose. Padrões de implante e cálculos de dose foram realizados usando os sistemas de implantes Patterson-Parker ou Quimby. Embora esses métodos de cálculo de dose tenham sido retirados com a implementação de sistemas de planejamento de tratamento utilizando computadores, a geometria básica do implante desses sistemas ainda é usada.


	
\section{Radioisótopos Utilizados em Braquiterapia}

\section{Aplicadores em Braquiterapia}

\section{O Processo da Braquiterapia}

\section{Workflow Em Braquiterapia}

\subsection*{Planejamento Inicial e Prescrição do Tratamento}

\subsection*{Pedido, Recebimento e Ensaios das Fontes}

\subsection*{Inserção de Aplicadores}

\subsection*{Imagens}

\subsection*{Planejamento de Tratamento}

\subsection*{QA Pré-Tratamento e Time-Out}

\subsection*{Entrega do Tratamento}

\subsection*{Recuperação de Fontes, Monitoramento Pós-Tratamento e Revisão de Registros}

\section{QA do Processo de Braquiterapia}

\section{Cálculos de Dose}


\section{Especificação e Report da Dose}


\section{Manuseio, Transporte, Armazenamento e Inventário das Fontes}




\bibliography{ref.bib}
\end{document}