\documentclass[12pt,a4paper]{article}
\usepackage[top=3cm, bottom=2cm, left=3cm, right=2cm]{geometry}
\usepackage[utf8]{inputenc}
% \usepackage[T1]{fontenc}
\usepackage{amsmath, amsfonts, amssymb}
\usepackage[brazil]{babel}
\usepackage{graphicx}
\usepackage{natbib}
\bibliographystyle{plainnat}
\bibpunct{[}{]}{,}{s}{}{}
\usepackage{color}
\usepackage{footnote}
\usepackage{setspace}
\usepackage{multirow}
\usepackage{fancyhdr}
\usepackage{leading}
\usepackage{indentfirst}
\usepackage{wrapfig}
\usepackage{float}
\renewcommand{\thefootnote}{\alph{footnote}}
\usepackage{url}
\pagestyle{fancy}
\fancyhf{}
\renewcommand{\headrulewidth}{0pt}
\rfoot{Página \thepage}
\title{Braquiterapia}
\author{Técnicas e Implantes\cite{devlin2015}}
\date{\textit{Dalila Mendonça}}
\begin{document}
	\maketitle
	\onehalfspacing



	\section{Classificações}

	Os procedimentos de Braquiterapia podem ser classificados de acordo com a técnica do implante, duração do tratamento, taxa de dose e o tipo de carregamento.

	\subsection{Classificações das Técnicas implantárias}

	Podem ser classificadas como:

	\begin{enumerate}
		\item Braquiterapia Intersticial: Técnica cujas fontes são colocadas diretamente em contato com o tecido.Exemplos:Braquiterapia de próstata, braquiterapia de cabeça e pescoço, braquiterapia de sarcomas, etc...
		
		\item Braquiterapia Intracavitária:  Procedimento onde as fontes são colocadas dentro de cavidades com o auxílio de um aplicador que entra em contato com o tecido que será tratado. Exemplos: Braquiterapia ginecológica (vagina e útero), etc...
		
		\item Braquiterapia Intracavitária:  Procedimento onde as fontes são colocadas dentro de cavidades com o auxílio de um aplicador que entra em contato com o tecido que será tratado. Exemplos: Braquiterapia ginecológica (vagina e útero), etc...
		
		\item Braquiterapia Intraluminal: Técnica que consiste na inserção das fontes com auxilio de catéteres em órgãos tubulares (lúmem). Exemplos: Esôfago e Traquéia.
		
		\item Braquiterapia Superficial: Consiste na inserção das fontes em plácas superficiais, moldes ou aplicadores que estarão em contato com a superfície do tecido para tratamento. Exemplos: Braquiterapia de pele e braquiterapia oftamológica.
		
		\item Braquiterapia Intravascular: Se trata da inserção das fontes em vasos sanguíneos.

	\end{enumerate}


	\subsection{Classifições quanto a duração do tratamento}

	Podem ser classificadas como:

	\begin{enumerate}
		\item Braquiterapia Permanente:  São os tratamentos no qual as fontes são permanentemente colocadas no paciente, ou seja, uma vez inseridas essas fontes não são removidas. Envolvem fontes de radiação com baixa atividade onde o tempo de meia-vida é muito menor comparado ao tempo de duração do implante (ao longo de todo o tempo de vida do paciente).
		
		\item Braquiterapia Temporária: São os tratamentos onde a fonte de radiação é implantada no paciente por um curto periodo de tempo e então a fonte é removida. Envolve fontes com alta atividade onde o tempo de meia-vida é muito maior que a duração do implante.
		
	\end{enumerate}


	\bibliography{ref.bib}

\end{document}